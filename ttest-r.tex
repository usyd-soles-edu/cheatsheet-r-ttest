% Options for packages loaded elsewhere
\PassOptionsToPackage{unicode}{hyperref}
\PassOptionsToPackage{hyphens}{url}
\PassOptionsToPackage{dvipsnames,svgnames,x11names}{xcolor}
%
\documentclass[
  letterpaper,
  DIV=11,
  numbers=noendperiod]{scrartcl}

\usepackage{amsmath,amssymb}
\usepackage{iftex}
\ifPDFTeX
  \usepackage[T1]{fontenc}
  \usepackage[utf8]{inputenc}
  \usepackage{textcomp} % provide euro and other symbols
\else % if luatex or xetex
  \usepackage{unicode-math}
  \defaultfontfeatures{Scale=MatchLowercase}
  \defaultfontfeatures[\rmfamily]{Ligatures=TeX,Scale=1}
\fi
\usepackage{lmodern}
\ifPDFTeX\else  
    % xetex/luatex font selection
\fi
% Use upquote if available, for straight quotes in verbatim environments
\IfFileExists{upquote.sty}{\usepackage{upquote}}{}
\IfFileExists{microtype.sty}{% use microtype if available
  \usepackage[]{microtype}
  \UseMicrotypeSet[protrusion]{basicmath} % disable protrusion for tt fonts
}{}
\makeatletter
\@ifundefined{KOMAClassName}{% if non-KOMA class
  \IfFileExists{parskip.sty}{%
    \usepackage{parskip}
  }{% else
    \setlength{\parindent}{0pt}
    \setlength{\parskip}{6pt plus 2pt minus 1pt}}
}{% if KOMA class
  \KOMAoptions{parskip=half}}
\makeatother
\usepackage{xcolor}
\setlength{\emergencystretch}{3em} % prevent overfull lines
\setcounter{secnumdepth}{-\maxdimen} % remove section numbering
% Make \paragraph and \subparagraph free-standing
\makeatletter
\ifx\paragraph\undefined\else
  \let\oldparagraph\paragraph
  \renewcommand{\paragraph}{
    \@ifstar
      \xxxParagraphStar
      \xxxParagraphNoStar
  }
  \newcommand{\xxxParagraphStar}[1]{\oldparagraph*{#1}\mbox{}}
  \newcommand{\xxxParagraphNoStar}[1]{\oldparagraph{#1}\mbox{}}
\fi
\ifx\subparagraph\undefined\else
  \let\oldsubparagraph\subparagraph
  \renewcommand{\subparagraph}{
    \@ifstar
      \xxxSubParagraphStar
      \xxxSubParagraphNoStar
  }
  \newcommand{\xxxSubParagraphStar}[1]{\oldsubparagraph*{#1}\mbox{}}
  \newcommand{\xxxSubParagraphNoStar}[1]{\oldsubparagraph{#1}\mbox{}}
\fi
\makeatother

\usepackage{color}
\usepackage{fancyvrb}
\newcommand{\VerbBar}{|}
\newcommand{\VERB}{\Verb[commandchars=\\\{\}]}
\DefineVerbatimEnvironment{Highlighting}{Verbatim}{commandchars=\\\{\}}
% Add ',fontsize=\small' for more characters per line
\usepackage{framed}
\definecolor{shadecolor}{RGB}{241,243,245}
\newenvironment{Shaded}{\begin{snugshade}}{\end{snugshade}}
\newcommand{\AlertTok}[1]{\textcolor[rgb]{0.68,0.00,0.00}{#1}}
\newcommand{\AnnotationTok}[1]{\textcolor[rgb]{0.37,0.37,0.37}{#1}}
\newcommand{\AttributeTok}[1]{\textcolor[rgb]{0.40,0.45,0.13}{#1}}
\newcommand{\BaseNTok}[1]{\textcolor[rgb]{0.68,0.00,0.00}{#1}}
\newcommand{\BuiltInTok}[1]{\textcolor[rgb]{0.00,0.23,0.31}{#1}}
\newcommand{\CharTok}[1]{\textcolor[rgb]{0.13,0.47,0.30}{#1}}
\newcommand{\CommentTok}[1]{\textcolor[rgb]{0.37,0.37,0.37}{#1}}
\newcommand{\CommentVarTok}[1]{\textcolor[rgb]{0.37,0.37,0.37}{\textit{#1}}}
\newcommand{\ConstantTok}[1]{\textcolor[rgb]{0.56,0.35,0.01}{#1}}
\newcommand{\ControlFlowTok}[1]{\textcolor[rgb]{0.00,0.23,0.31}{\textbf{#1}}}
\newcommand{\DataTypeTok}[1]{\textcolor[rgb]{0.68,0.00,0.00}{#1}}
\newcommand{\DecValTok}[1]{\textcolor[rgb]{0.68,0.00,0.00}{#1}}
\newcommand{\DocumentationTok}[1]{\textcolor[rgb]{0.37,0.37,0.37}{\textit{#1}}}
\newcommand{\ErrorTok}[1]{\textcolor[rgb]{0.68,0.00,0.00}{#1}}
\newcommand{\ExtensionTok}[1]{\textcolor[rgb]{0.00,0.23,0.31}{#1}}
\newcommand{\FloatTok}[1]{\textcolor[rgb]{0.68,0.00,0.00}{#1}}
\newcommand{\FunctionTok}[1]{\textcolor[rgb]{0.28,0.35,0.67}{#1}}
\newcommand{\ImportTok}[1]{\textcolor[rgb]{0.00,0.46,0.62}{#1}}
\newcommand{\InformationTok}[1]{\textcolor[rgb]{0.37,0.37,0.37}{#1}}
\newcommand{\KeywordTok}[1]{\textcolor[rgb]{0.00,0.23,0.31}{\textbf{#1}}}
\newcommand{\NormalTok}[1]{\textcolor[rgb]{0.00,0.23,0.31}{#1}}
\newcommand{\OperatorTok}[1]{\textcolor[rgb]{0.37,0.37,0.37}{#1}}
\newcommand{\OtherTok}[1]{\textcolor[rgb]{0.00,0.23,0.31}{#1}}
\newcommand{\PreprocessorTok}[1]{\textcolor[rgb]{0.68,0.00,0.00}{#1}}
\newcommand{\RegionMarkerTok}[1]{\textcolor[rgb]{0.00,0.23,0.31}{#1}}
\newcommand{\SpecialCharTok}[1]{\textcolor[rgb]{0.37,0.37,0.37}{#1}}
\newcommand{\SpecialStringTok}[1]{\textcolor[rgb]{0.13,0.47,0.30}{#1}}
\newcommand{\StringTok}[1]{\textcolor[rgb]{0.13,0.47,0.30}{#1}}
\newcommand{\VariableTok}[1]{\textcolor[rgb]{0.07,0.07,0.07}{#1}}
\newcommand{\VerbatimStringTok}[1]{\textcolor[rgb]{0.13,0.47,0.30}{#1}}
\newcommand{\WarningTok}[1]{\textcolor[rgb]{0.37,0.37,0.37}{\textit{#1}}}

\providecommand{\tightlist}{%
  \setlength{\itemsep}{0pt}\setlength{\parskip}{0pt}}\usepackage{longtable,booktabs,array}
\usepackage{calc} % for calculating minipage widths
% Correct order of tables after \paragraph or \subparagraph
\usepackage{etoolbox}
\makeatletter
\patchcmd\longtable{\par}{\if@noskipsec\mbox{}\fi\par}{}{}
\makeatother
% Allow footnotes in longtable head/foot
\IfFileExists{footnotehyper.sty}{\usepackage{footnotehyper}}{\usepackage{footnote}}
\makesavenoteenv{longtable}
\usepackage{graphicx}
\makeatletter
\def\maxwidth{\ifdim\Gin@nat@width>\linewidth\linewidth\else\Gin@nat@width\fi}
\def\maxheight{\ifdim\Gin@nat@height>\textheight\textheight\else\Gin@nat@height\fi}
\makeatother
% Scale images if necessary, so that they will not overflow the page
% margins by default, and it is still possible to overwrite the defaults
% using explicit options in \includegraphics[width, height, ...]{}
\setkeys{Gin}{width=\maxwidth,height=\maxheight,keepaspectratio}
% Set default figure placement to htbp
\makeatletter
\def\fps@figure{htbp}
\makeatother

\usepackage{booktabs}
\usepackage{caption}
\usepackage{longtable}
\usepackage{colortbl}
\usepackage{array}
\usepackage{anyfontsize}
\usepackage{multirow}
\KOMAoption{captions}{tableheading}
\makeatletter
\@ifpackageloaded{tcolorbox}{}{\usepackage[skins,breakable]{tcolorbox}}
\@ifpackageloaded{fontawesome5}{}{\usepackage{fontawesome5}}
\definecolor{quarto-callout-color}{HTML}{909090}
\definecolor{quarto-callout-note-color}{HTML}{0758E5}
\definecolor{quarto-callout-important-color}{HTML}{CC1914}
\definecolor{quarto-callout-warning-color}{HTML}{EB9113}
\definecolor{quarto-callout-tip-color}{HTML}{00A047}
\definecolor{quarto-callout-caution-color}{HTML}{FC5300}
\definecolor{quarto-callout-color-frame}{HTML}{acacac}
\definecolor{quarto-callout-note-color-frame}{HTML}{4582ec}
\definecolor{quarto-callout-important-color-frame}{HTML}{d9534f}
\definecolor{quarto-callout-warning-color-frame}{HTML}{f0ad4e}
\definecolor{quarto-callout-tip-color-frame}{HTML}{02b875}
\definecolor{quarto-callout-caution-color-frame}{HTML}{fd7e14}
\makeatother
\makeatletter
\@ifpackageloaded{caption}{}{\usepackage{caption}}
\AtBeginDocument{%
\ifdefined\contentsname
  \renewcommand*\contentsname{Table of contents}
\else
  \newcommand\contentsname{Table of contents}
\fi
\ifdefined\listfigurename
  \renewcommand*\listfigurename{List of Figures}
\else
  \newcommand\listfigurename{List of Figures}
\fi
\ifdefined\listtablename
  \renewcommand*\listtablename{List of Tables}
\else
  \newcommand\listtablename{List of Tables}
\fi
\ifdefined\figurename
  \renewcommand*\figurename{Figure}
\else
  \newcommand\figurename{Figure}
\fi
\ifdefined\tablename
  \renewcommand*\tablename{Table}
\else
  \newcommand\tablename{Table}
\fi
}
\@ifpackageloaded{float}{}{\usepackage{float}}
\floatstyle{ruled}
\@ifundefined{c@chapter}{\newfloat{codelisting}{h}{lop}}{\newfloat{codelisting}{h}{lop}[chapter]}
\floatname{codelisting}{Listing}
\newcommand*\listoflistings{\listof{codelisting}{List of Listings}}
\makeatother
\makeatletter
\makeatother
\makeatletter
\@ifpackageloaded{caption}{}{\usepackage{caption}}
\@ifpackageloaded{subcaption}{}{\usepackage{subcaption}}
\makeatother
\ifLuaTeX
  \usepackage{selnolig}  % disable illegal ligatures
\fi
\usepackage{bookmark}

\IfFileExists{xurl.sty}{\usepackage{xurl}}{} % add URL line breaks if available
\urlstyle{same} % disable monospaced font for URLs
\hypersetup{
  pdftitle={One-sample t-test in R},
  colorlinks=true,
  linkcolor={blue},
  filecolor={Maroon},
  citecolor={Blue},
  urlcolor={Blue},
  pdfcreator={LaTeX via pandoc}}

\title{One-sample \emph{t}-test in R}
\usepackage{etoolbox}
\makeatletter
\providecommand{\subtitle}[1]{% add subtitle to \maketitle
  \apptocmd{\@title}{\par {\large #1 \par}}{}{}
}
\makeatother
\subtitle{Cheatsheet}
\author{}
\date{2024-08-10}

\begin{document}
\maketitle

\begin{tcolorbox}[enhanced jigsaw, opacitybacktitle=0.6, toprule=.15mm, colbacktitle=quarto-callout-note-color!10!white, colback=white, coltitle=black, title=\textcolor{quarto-callout-note-color}{\faInfo}\hspace{0.5em}{License}, bottomrule=.15mm, left=2mm, breakable, bottomtitle=1mm, opacityback=0, rightrule=.15mm, titlerule=0mm, arc=.35mm, toptitle=1mm, leftrule=.75mm, colframe=quarto-callout-note-color-frame]

This work was developed using resources that are available under a
\href{https://creativecommons.org/licenses/by/4.0/}{Creative Commons
Attribution 4.0 International License}, made available on the
\href{https://github.com/usyd-soles-edu}{SOLES Open Educational
Resources} repository by the School of Life and Environmental Sciences,
The University of Sydney.

\end{tcolorbox}

\begin{tcolorbox}[enhanced jigsaw, opacitybacktitle=0.6, toprule=.15mm, colbacktitle=quarto-callout-note-color!10!white, colback=white, coltitle=black, title=\textcolor{quarto-callout-note-color}{\faInfo}\hspace{0.5em}{Assumed knowledge}, bottomrule=.15mm, left=2mm, breakable, bottomtitle=1mm, opacityback=0, rightrule=.15mm, titlerule=0mm, arc=.35mm, toptitle=1mm, leftrule=.75mm, colframe=quarto-callout-note-color-frame]

\begin{itemize}
\tightlist
\item
  You know how to install and load packages in R.
\item
  You know how to import data into R.
\item
  You recognise data frames and vectors.
\end{itemize}

\end{tcolorbox}

\begin{tcolorbox}[enhanced jigsaw, opacitybacktitle=0.6, toprule=.15mm, colbacktitle=quarto-callout-important-color!10!white, colback=white, coltitle=black, title=\textcolor{quarto-callout-important-color}{\faExclamation}\hspace{0.5em}{Data structure}, bottomrule=.15mm, left=2mm, breakable, bottomtitle=1mm, opacityback=0, rightrule=.15mm, titlerule=0mm, arc=.35mm, toptitle=1mm, leftrule=.75mm, colframe=quarto-callout-important-color-frame]

The data should be in a \textbf{long format} (also known as tidy data),
where each row is an observation and each column is a variable
(Figure~\ref{fig-format}). If your data is not already structured this
way, reshape it manually in a spreadsheet program or in R using the
\texttt{pivot\_longer()} function from the \texttt{tidyr} package.

\begin{figure}[H]

\begin{minipage}{0.44\linewidth}

\begingroup
\fontsize{12.0pt}{14.4pt}\selectfont
\begin{longtable*}{lr}
\toprule
Sex & BW \\ 
\midrule\addlinespace[2.5pt]
F & 2.15 \\ 
M & 2.55 \\ 
F & 2.95 \\ 
F & 2.70 \\ 
M & 2.20 \\ 
F & 1.85 \\ 
M & 2.55 \\ 
M & 2.60 \\ 
\bottomrule
\end{longtable*}
\endgroup

\end{minipage}%
%
\begin{minipage}{0.11\linewidth}
~\end{minipage}%
%
\begin{minipage}{0.44\linewidth}

\begingroup
\fontsize{12.0pt}{14.4pt}\selectfont
\begin{longtable*}{rr}
\toprule
F & M \\ 
\midrule\addlinespace[2.5pt]
2.15 & 2.55 \\ 
2.95 & 2.20 \\ 
2.70 & 2.55 \\ 
1.85 & 2.60 \\ 
\bottomrule
\end{longtable*}
\endgroup

\end{minipage}%

\caption{\label{fig-format}Data should be in long format (left) where
each row is an observation and each column is a variable. This is the
preferred format for most statistical software. Wide format (right) is
also common, but may require additional steps to analyse or visualise in
some instances.}

\end{figure}%

\end{tcolorbox}

\begin{tcolorbox}[enhanced jigsaw, opacitybacktitle=0.6, toprule=.15mm, colbacktitle=quarto-callout-important-color!10!white, colback=white, coltitle=black, title=\textcolor{quarto-callout-important-color}{\faExclamation}\hspace{0.5em}{Data}, bottomrule=.15mm, left=2mm, breakable, bottomtitle=1mm, opacityback=0, rightrule=.15mm, titlerule=0mm, arc=.35mm, toptitle=1mm, leftrule=.75mm, colframe=quarto-callout-important-color-frame]

For this cheatsheet we will use the entire possums dataset used in
\href{https://www.sydney.edu.au/units/BIOL2022}{BIOL2022} labs.

\end{tcolorbox}

\subsection{About}\label{about}

The one-sample \emph{t}-test is used to determine whether the mean of a
single sample \(y\) is significantly different from a known or
hypothesised population mean (\(\mu\)). \textbf{Examples}:

\begin{itemize}
\tightlist
\item
  Is the mean weight of canned tuna significantly different from what
  was stated on the label (400 g)?
\item
  Is the mean height of a sample of male students significantly
  different from the national average height (175.6 cm)?
\item
  Is the mean number of kittens in a litter significantly different from
  4?
\end{itemize}

\subsection{Modelling}\label{modelling}

\begin{quote}
Is the mean \textbf{body weight} of possums (\texttt{BW}) significantly
different from 3.5 kg?
\end{quote}

The {\textbf{simplified model}} for the mathematically-adverse
individual is \[\color{olive}\text{body weight} \sim 3.5\] which
translates to ``the body weight of possums is \emph{around} 3.5 kg''.
The {\textbf{statistical model}} is
\[\color{red}\text{body weight} = \beta_0 + \epsilon\] where \(\beta_0\)
is the hypothesised population mean and \(\epsilon\) is the error term.

\subsubsection{Preparing the data}\label{preparing-the-data}

Extract \textbf{only} the variable of interest from the dataset using
\texttt{select()} from the \texttt{dplyr} package -- \texttt{BW}. Assign
the variable to a new object -- \texttt{bw} in this case.

\begin{Shaded}
\begin{Highlighting}[]
\FunctionTok{library}\NormalTok{(dplyr)}
\FunctionTok{library}\NormalTok{(readxl)}
\NormalTok{possums }\OtherTok{\textless{}{-}} \FunctionTok{read\_excel}\NormalTok{(}\StringTok{"possums.xlsx"}\NormalTok{, }\AttributeTok{sheet =} \DecValTok{2}\NormalTok{) }\CommentTok{\# import}
\NormalTok{bw }\OtherTok{\textless{}{-}} \FunctionTok{select}\NormalTok{(possums, BW) }\CommentTok{\# select variable}
\end{Highlighting}
\end{Shaded}

Your own data should be in a similar format.

\subsection{Analytical approaches}\label{analytical-approaches}

The traditional approach to the one-sample \emph{t}-test is to use the
\texttt{t.test()} function in R, while the modern approach is to use a
general linear model (GLM) with the \texttt{lm()} or \texttt{glm()}
functions.

\subsection{\texorpdfstring{\texttt{t.test()}
function}{t.test() function}}

\subsubsection{Methods reporting}\label{methods-reporting}

\begin{quote}
A one-sample \emph{t}-test was used to determine whether the mean body
weight of possums was significantly different from 3.5 kg. This was
computed using the \texttt{t.test()} function in R version 4.4.0 (R Core
Team, 2024).
\end{quote}

\subsubsection{Perform the analysis}\label{perform-the-analysis}

\begin{Shaded}
\begin{Highlighting}[]
\FunctionTok{t.test}\NormalTok{(bw, }\AttributeTok{mu =} \FloatTok{3.5}\NormalTok{)}
\end{Highlighting}
\end{Shaded}

\subsubsection{Check assumption(s)}\label{check-assumptions}

\paragraph{Normality}\label{normality}

Any combination of one or more of the following checks can be used to
assess normality:

\begin{itemize}
\tightlist
\item
  \textbf{Histogram}: \texttt{hist(bw\$BW)}
\item
  \textbf{Q-Q plot}: \texttt{qqnorm(bw\$BW)}
\item
  \textbf{Shapiro-Wilk test}: \texttt{shapiro.test(bw\$BW)}
\end{itemize}

Include the appropriate description in your methods section.

\begin{quote}
The normality of body weight was assessed using {[}insert method(s){]}.
\end{quote}

\subsubsection{How to report results}\label{how-to-report-results}

\begin{quote}
The mean body weight of possums was significantly different from 3.5 kg
(t\textsubscript{19} = -10.3, 95\% CI {[}2.3, 2.7{]}, p \textless{}
0.001).
\end{quote}

\subsection{\texorpdfstring{\texttt{lm()} function}{lm() function}}

\subsubsection{Methods reporting}\label{methods-reporting-1}

\begin{quote}
A general linear model was used to determine whether the mean body
weight of possums was significantly different from 3.5 kg. This was
computed using the \texttt{lm()} function in R version 4.4.0 (R Core
Team, 2024).
\end{quote}

\subsubsection{Perform the analysis}\label{perform-the-analysis-1}

For a one-sample \emph{t}-test, the formula needs to be specified as
\texttt{y\ -\ µ\ \textasciitilde{}\ 1} where \texttt{y} is the variable
of interest and µ is the hypothesised value that is being tested. The
\texttt{1} indicates that the model has an intercept only i.e.~we are
testing whether the mean difference is significantly different from 0.

\begin{Shaded}
\begin{Highlighting}[]
\NormalTok{fit }\OtherTok{\textless{}{-}} \FunctionTok{lm}\NormalTok{((BW }\SpecialCharTok{{-}} \FloatTok{3.5}\NormalTok{) }\SpecialCharTok{\textasciitilde{}} \DecValTok{1}\NormalTok{, }\AttributeTok{data =}\NormalTok{ bw)}
\FunctionTok{summary}\NormalTok{(fit)}
\end{Highlighting}
\end{Shaded}

\subsubsection{Check assumption(s)}\label{check-assumptions-1}

\paragraph{Normality}\label{normality-1}

With a GLM, normality can be assessed using the residuals of the model.
The following checks can be used:

\begin{itemize}
\tightlist
\item
  \textbf{Histogram}: \texttt{hist(residuals(fit))}
\item
  \textbf{Q-Q plot}: \texttt{qqnorm(residuals(fit))}
\item
  \textbf{Shapiro-Wilk test}: \texttt{shapiro.test(residuals(fit))}
\end{itemize}

\subsubsection{How to report results}\label{how-to-report-results-1}

\begin{quote}
There is evidence to suggest that the mean body weight of possums was
significantly different from 3.5 kg (GLM, t\textsubscript{19} = -10.3, p
\textless{} 0.001).
\end{quote}

\subsection{Exercise(s)}\label{exercises}

Download the penguins dataset (from below if you are reading this in
HTML), or load the dataset from the \texttt{palmerpenguins} package.
Perform a one-sample \emph{t}-test to determine whether the mean flipper
length of penguins is significantly different from 200 mm.



\end{document}
